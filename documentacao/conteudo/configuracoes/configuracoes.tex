
% --- 
% CONFIGURAÇÕES DE PACOTES
% --- 

% ---
% Configuração de pseudocodigo
% ---

% ---
%Configuração do pacote tabularx
\renewcommand\tabularxcolumn[1]{m{#1}}% for vertical centering text in X column
\newcolumntype{Y}{>{\centering\arraybackslash}X}
\newcolumntype{N}{>{\centering\arraybackslash}m{6cm}}
\newcolumntype{G}{>{\centering\arraybackslash}m{20cm}}

% Configurações do pacote backref
% Usado sem a opção hyperpageref de backref
\renewcommand{\backrefpagesname}{Citado na(s) página(s):~}
% Texto padrão antes do número das páginas
\renewcommand{\backref}{}
% Define os textos da citação
\renewcommand*{\backrefalt}[4]{
	\ifcase #1 %
		Nenhuma citação no texto.%
	\or
		Citado na página #2.%
	\else
		Citado #1 vezes nas páginas #2.%
	\fi}%
% ---



% ---
% Informações de dados para CAPA e FOLHA DE ROSTO
% ---
\titulo{Desenvolvimento de uma Inteligência Artificial para aprender a jogar jogos em Allegro}
\autor{Arthur de Senna Rocha}
\local{Brasil}
\data{\today}
\orientador{Pedro Olmo Stancioli Vaz De Melo}
%\coorientador{Equipe \abnTeX}
\instituicao{%
  Universidade Federal de Minas Gerais -- UFMG
  \par
  Escola de Engenharia
  \par
  Engenharia de Sistemas}
\tipotrabalho{Trabalho de Conclusão de Curso}
% O preambulo deve conter o tipo do trabalho, o objetivo, 
% o nome da instituição e a área de concentração 
\preambulo{Trabalho de conclusão de curso apresentado a Engenharia de Sistemas, como parte dos requisitos necessário à obtenção de título de Bacharelado em Engenheiro de Sistemas.
}
% ---


% ---
% Configurações de aparência do PDF final

% alterando o aspecto da cor azul
\definecolor{blue}{RGB}{41,5,195}

% informações do PDF
\makeatletter
\hypersetup{
     	%pagebackref=true,
		pdftitle={\@title}, 
		pdfauthor={\@author},
    	pdfsubject={\@title},
	    pdfcreator={\@author},
		pdfkeywords={Deep Learning}{Allegro}{Inteligência Artificial}{Jogos Digitais}{Machine Learning}{Trabalho de conclusão de curso}{TCC}, 
		colorlinks=true,       		% false: boxed links; true: colored links
    	linkcolor=blue,          	% color of internal links
    	citecolor=blue,        		% color of links to bibliography
    	filecolor=magenta,      		% color of file links
		urlcolor=blue,
		bookmarksdepth=4
}
\makeatother
% --- 

% ---
% Posiciona figuras e tabelas no topo da página quando adicionadas sozinhas
% em um página em branco. Ver https://github.com/abntex/abntex2/issues/170
\makeatletter
\setlength{\@fptop}{5pt} % Set distance from top of page to first float
\makeatother
% ---

% ---
% Possibilita criação de Quadros e Lista de quadros.
% Ver https://github.com/abntex/abntex2/issues/176
%
\newcommand{\quadroname}{Quadro}
\newcommand{\listofquadrosname}{Lista de quadros}

\newfloat[chapter]{quadro}{loq}{\quadroname}
\newlistof{listofquadros}{loq}{\listofquadrosname}
\newlistentry{quadro}{loq}{0}

% configurações para atender às regras da ABNT
\setfloatadjustment{quadro}{\centering}
\counterwithout{quadro}{chapter}
\renewcommand{\cftquadroname}{\quadroname\space} 
\renewcommand*{\cftquadroaftersnum}{\hfill--\hfill}

\setfloatlocations{quadro}{hbtp} % Ver https://github.com/abntex/abntex2/issues/176
% ---

% --- 
% Espaçamentos entre linhas e parágrafos 
% --- 

% O tamanho do parágrafo é dado por:
\setlength{\parindent}{1.3cm}

% Controle do espaçamento entre um parágrafo e outro:
\setlength{\parskip}{0.2cm}  % tente também \onelineskip
