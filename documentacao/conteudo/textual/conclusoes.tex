% ----------------------------------------------------------
% Contextualização em Humanidades
% ----------------------------------------------------------
\chapter{Conclusões}
\label{chap:conclusoes}
% ----------------------------------------------------------

A inteligência artificial e o aprendizado de máquina são ferramentas muito poderosas e com inúmeras aplicações práticas. A IA busca fornecer softwares que sejam capazes de realizar atividades como seres humanos para automatizar e otimizar o trabalho de rotina. Diante do grande potencial da IA, além do crescimento exponencial de pesquisa que a área vêm sofrendo, grandes empresas no mercado estão investindo na tecnologia, seja para propor novos serviços ou aprimorar produtos existentes e garantir uma vantagem competitiva no mercado.

 % Reproduzir e comparar o trabalhos existentes existente e julgar com precisão as melhorias oferecidas por novos métodos é vital para sustentar esse progresso. 
% O crescente investimento em pesquisa vem gerando IA cada vez mais

Situações do mundo real são muitas vezes complexas e apresentam problemas com um número muito grande de variáveis, o que dificulta a solução utilizando algoritmos de otimização tradicionais. Treinar um agente em jogos digitais para superar os jogadores humanos e otimizar sua pontuação pode nos ensinar como otimizar processos variados com múltiplas aplicações. Uma vez que se tenha uma IA que possa aprender a jogar e a otimizar estratégias para maximizar a pontuação de um jogo, pode-se facilmente implementar um jogo que simule uma situação real e aplicar o sistema para que este encontre a melhor resposta ou solução para um dado problema. Com isso em mente, o problema proposto nesse trabalho é o de implementar uma IA que, utilizando algoritmos de \textit{deep reinforcement learning}, seja capaz de aprender e desenvolver estratégias para jogar diferentes jogos digitais. 

A IA proposta deverá ser genérica, ou seja, capaz de aprender a jogar diferentes jogos, desde que se tenha acesso ao codigo fonte e que sejam implementados em \textit{Allegro}. Para auxiliar na implementação do sistema será utilizado um \textit{Allegro Learning Enviroment}, plataforma que irá facilitar a implementação da ferramenta para o treinamento do agente. 

Diante disso, utilizando técnicas de DRL existentes, espera-se produzir uma IA que seja flexível e que possa ser adaptada para diferentes cenários. Neste sentido, espera-se uma IA que seja genérica e capaz de ser treinada para diversos jogos. Por fim, será feita uma análise crítica dos resultados e uma comparação dos mesmos com trabalhos semelhantes realizados por outras entidades.
\clearpage
\section{Proposta de Continuidade} % (fold)
\label{sec:proposta_de_continuidade}

Este trabalho tem como continuidade o desenvolvimento do Trabalho de Conclusão de Curso II, onde haverá um maior detalhamento sobre a modelagem matemática do problema, além de especificadas as decisões de implementação da ferramenta elaborada, bem como uma análise dos resultados.

A abordagem, além do que já foi exposto, consistirá na implementação da rede neural proposta, utilizando os algoritmos DRL mencionados para o treinamento do agente. Também serão implementados (se necessário), diferentes jogos em \textit{Allegro} para a validação do sistema. Por fim, o agente será treinado em jogos simples e de baixa complexidade e serão apresentados os resultados obtidos para avaliar o potencial do sistema. Em trabalhos futuros, a ferramenta poderá também ser utilizada para o treinamento em jogos de diferentes complexidades, ampliando ainda mais o alcance do sistema.

% section proposta_de_continuidade (end)

