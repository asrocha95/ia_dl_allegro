% ----------------------------------------------------------
% Abordagem proposta 
% ----------------------------------------------------------
\chapter{Abordagem Proposta}
% ----------------------------------------------------------

No presente trabalho é proposto a implementação de uma inteligência artificial que, utilizando um algoritmo de \textit{deep reinforcement learning} (DRL), seja capaz de aprender a jogar diferentes jogos e desenvolver estratégias para maximizar sua pontuação. O DRL é uma abordagem do \textit{deep learning} que, em contraste à abordagens mais comuns como o aprendizado supervisionado e não supervisionado, utiliza as técnicas de aprendizagem por reforço para treinar o agente.

O \textit{Allegro} é uma biblioteca multiplataforma destinada principalmente a jogos de vídeo e programação multimídia. A biblioteca fornece rotinas de baixo nível comumente necessárias na programação de jogos, como a criação de janelas, aceitação de entrada do usuário, carregamento de dados, desenho de imagens, reprodução de sons etc \cite{allegro}.

Para auxiliar a implementação do sistema, será utilizado uma ferramenta para desenvolvimento de inteligência artificial em jogos allegro \cite{silva:amb-jd-allegro}, que oferece um ambiente facilitador ao estudo de soluções de IA aplicada em jogos. Essa ferramenta fornece funcionalidades como a exportação dos comandos básicos de um jogo, que precisam ser passados para o agente para que o mesmo tenha conhecimento dos limites físicos do ambiente no qual está inserido. Essa ferramenta permite que o pesquisador não fique limitado a um jogo existente, mas possa usar qualquer jogo que ele tenha acesso ao código fonte e feito em \textit{Allegro}.
