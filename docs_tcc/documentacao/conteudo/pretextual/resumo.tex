
% resumo em português
\setlength{\absparsep}{18pt} % ajusta o espaçamento dos parágrafos do resumo
\begin{resumo}
	% A inteligência artificial (IA) vem ganhando manchetes no mundo todo.
	%sendo anunciada tanto como uma salvação econômica quanto como precursora de desintegração social.
	O uso de inteligência artificial (IA) e de algoritmos de \textit{machine learning}  possibilita que máquinas aprendam com experiências, se ajustem à novas entradas de dados e performem tarefas como seres humanos. Com essas tecnologias, os computadores podem ser treinados para cumprir tarefas específicas ao processar grandes quantidades de dados e reconhecer padrões nesses dados.
	% A IA adiciona inteligência a produtos existentes. Na maioria dos casos, a inteligência artificial não é vendida como uma aplicação individual. Pelo contrário, produtos já existentes são aprimorados com funcionalidades de IA, de maneira parecida como a Siri foi adicionada aos produtos da \textit{Apple}. Automação, plataformas de conversa, robôs e aparelhos inteligentes podem ser combinados com grandes quantidades de dados para aprimorar diversas tecnologias para casa e escritório, de inteligência em segurança à análise de investimentos.
	% A maioria dos exemplos de IA sobre os quais se ouve falar hoje – de computadores mestres em xadrez a carros autônomos – dependem de \textit{deep learning} e processamento de linguagem natural (PNL). Treinar um agente para superar os jogadores humanos e otimizar sua performance pode nos ensinar como otimizar diferentes processos em uma grande variedade de situações. Foi o que o \textit{DeepMind} do Google fez com seu popular \textit{AlphaGo} e seu sucessor \textit{AlphaZero}, vencendo os campeões mundiais em Go, xadrez e shogi, e obtendo resultados de performance nunca antes vistos.
	O presente trabalho se propõe a desenvolver uma IA capaz de aprender a jogar diferentes jogos, desde que se tenha acesso ao código fonte e feito em Allegro. Para isso, será implementado um algoritmo de \textit{Deep Reinforcement Learning}, abordagem que consiste em fornecer ao sistema parâmetros relacionados ao seu estado e uma recompensa positiva ou negativa com base em suas ações. Nenhuma regra sobre o jogo é dada e, inicialmente, a IA não tem informações sobre o que precisa fazer. A única informação passada para a IA são os comandos básicos do jogo. O objetivo do sistema é descobrir e elaborar uma estratégia para maximizar a pontuação - ou a recompensa. Diferente de muitas IAs que focam na solução de um único problema, a proposta deste projeto é elaborar uma IA que seja genérica e capaz solucionar e elaborar estratégias para uma variedade de situações diferentes.

	\vspace{\onelineskip}

	\noindent	
	\textbf{Palavras-chave}: deep learning, allegro, inteligência artificial, jogos digitais, machine learning.
\end{resumo}

% % resumo em inglês
\begin{resumo}[Abstract]
 \begin{otherlanguage*}{english}
	The use of Artificial intelligence (AI) and machine learning algorithms enables computers to learn from experience, adjust to new data inputs, and perform tasks as human beings. With these technologies, computers can be trained to perform specific tasks by processing large amounts of data and recognizing patterns in that data.
	The present work aims to develop an AI capable of learning how to play different games, provided that it has access to the source code and the game is made in Allegro. For this, a Deep Reinforcement Learning algorithm will be implemented, which provides the system with parameters related to its state and a positive or negative reward based on its actions. No rules about the game are given and initially, the AI has no information on what it needs to do. The only information passed to AI is the basic commands of the game. The goal of the system is to discover and devise a strategy to maximize its score - or the reward. Unlike many AIs that focus on solving a single problem, the purpose of this project is to design a generic AI that can solve and develop a strategy for a variety of different situations.

   	\vspace{\onelineskip}

   	\noindent 
   	\textbf{Keywords}: deep learning, allegro, artificial intelligence, video games, machine learning.
 \end{otherlanguage*}
\end{resumo}

% % resumo em francês 
% \begin{resumo}[Résumé]
%  \begin{otherlanguage*}{french}
%     Il s'agit d'un résumé en français.
 
%    \textbf{Mots-clés}: latex. abntex. publication de textes.
%  \end{otherlanguage*}
% \end{resumo}

% % resumo em espanhol
% \begin{resumo}[Resumen]
%  \begin{otherlanguage*}{spanish}
%    Este es el resumen en español.
  
%    \textbf{Palabras clave}: latex. abntex. publicación de textos.
%  \end{otherlanguage*}
% \end{resumo}