% ----------------------------------------------------------
% Contextualização em Humanidades
% ----------------------------------------------------------
\chapter{Comentários finais TCC I e Proposta para o TCC II}
\label{chap:conclusoes_tcc1}
% ----------------------------------------------------------

O problema proposto nesse trabalho é o de implementar uma IA que, utilizando algoritmos de \textit{deep reinforcement learning}, seja capaz de aprender e desenvolver estratégias para jogar diferentes jogos digitais. A IA proposta deverá ser genérica, ou seja, capaz de aprender a jogar diferentes jogos, desde que se tenha acesso ao codigo fonte e que sejam implementados em \textit{Allegro}. Para auxiliar na implementação do sistema será utilizado um \textit{Allegro Learning Enviroment}, plataforma que irá facilitar a implementação da ferramenta para o treinamento do agente. 

Diante disso, utilizando técnicas de DRL existentes, espera-se produzir uma IA que seja flexível e que possa ser adaptada para diferentes cenários. Neste sentido, espera-se uma IA que seja genérica e capaz de ser treinada para diversos jogos. Por fim, será feita uma análise crítica dos resultados e uma comparação dos mesmos com trabalhos semelhantes realizados por outras entidades.

\section{Proposta TCC 2} % (fold)
\label{sec:proposta_de_continuidade}

Este trabalho tem como continuidade o desenvolvimento do Trabalho de Conclusão de Curso II, onde haverá um maior detalhamento sobre a modelagem matemática do problema, além de especificadas as decisões de implementação da ferramenta elaborada, bem como uma análise dos resultados.

A abordagem, além do que já foi exposto, consistirá na implementação da rede neural proposta, utilizando os algoritmos DRL mencionados para o treinamento do agente. Também serão implementados (se necessário), diferentes jogos em \textit{Allegro} para a validação do sistema. Por fim, o agente será treinado em jogos simples e de baixa complexidade e serão apresentados os resultados obtidos para avaliar o potencial do sistema. Em trabalhos futuros, a ferramenta poderá também ser utilizada para o treinamento em jogos de diferentes complexidades, ampliando ainda mais o alcance do sistema.

\clearpage
 \section{Cronograma TCC 2}

\begin{table}[h]
	\begin{center}
		\begin{tabularx}{\textwidth}{|N|Y|Y|Y|Y|}
			\hline
			\multicolumn{1}{|N|}{Atividades}&\multicolumn{4}{|c|}{\centering{Meses}}\\
			\hline
			&Agosto&Setembro&Outubro&Novembro\\
			\hline
			Levantamento bibliográfico&X&&&\\
			\hline
			Pesquisa e implementação da rede neural proposta do trabalho&X&&&\\
			\hline
			Aplicação do estudo realizado no assunto do TCC a ser desenvolvido&X&&&\\
			\hline
			Entrega da Visão Geral do Trabalho&21/08/2020&&&\\
			\hline
			Desenvolvimento do trabalho e implementação da rede neural proposta do trabalho&X&X&&\\
			\hline
			Elaboração do corpo principal do TCC&&X&X&\\
			\hline
			Entrega do FomulárioPonto de Controle&&11/09/2020&&\\
			\hline
			Marcação da defesa&&25/09/2020&&\\
			\hline
			Ajustes finais e conclusão to trabalho implementado&&&X&\\
			\hline
			% Elaboração da conclusão e fechamento do TCC&&&X&\\
			% \hline
			Emissão da versão inicial do TCC&&&17/10/2020&\\
			\hline
			Preparação do material referente à apresentação do TCC&&&X&\\
			\hline
			Apresentação oral para banca examinadora&&&Semana de 26 a 30/10&\\
			\hline
			Ajuste no material relativo ao trabalho escrito&&&&06/11/2020\\
			\hline
		\end{tabularx}
	\end{center}
\end{table}

% section proposta_de_continuidade (end)

